\begin{recipe}{Kosheri}{Serves 5}{45min}
\Ing{4 tbsp olive oil}
\Ing{2 garlic cloves, crushed}
\Ing{2 hot red chillies, seeded and finely diced}
\Ing{8 ripe tomatoes, chopped (tinned are fine)}
\Ing{370ml water}
\Ing{4 tbsp cider vinegar}
\Ing{3 tsp salt}
\Ing{2 tsp ground cumin}
\Ing{20g coriander leaves, chopped}
To make the sauce, heat the olive oil in a saucepan, add the garlic and chills and fry for 2 minutes. Add the chopped tomatoes, water, vinegar, salt, and cumin. Bring to a boil, then lower the heat and simmer for 20 minutes, until slightly thickened. Remove the sauce from the heat, stir in the cilantro, and then taste. See if you want to add any salt, pepper, or extra cilantro. Keep hot or leave to cool; both ways will work with the hot kosheri. Just remember to adjust the seasoning again when cold.
\Ing{300g green lentils}
Place the lentils in a large sieve and wash them under a cold running tap. Transfer to a large saucepan, cover with plenty of cold water, and bring to a boil. Lower the heat and simmer for 25 minutes. The lentils should be tender but far from mushy. Drain in a colander and set aside.

\Ing{200g basmati rice}
\Ing{40g unsalted butter}
\Ing{50g vermicelli noodles, broken into 4cm pieces}
\Ing{400ml chicken stock or water}
\Ing{\frac{1}{2} tsp grated nutmeg}
\Ing{1 & \frac{1}{2}tsp ground cinnamon}
\Ing{1 & \frac{1}{2}tsp salt}
\Ing{\frac{1}{2}tsp black pepper}
In a large bowl, cover the rice with cold water, wash, and then drain well. Melt the butter in a large saucepan over medium heat. Add the raw vermicelli. Stir, and continue frying and stirring until the vermicelli is golden brown. Add the drained rice and mix well until it is well coated in the butter. Now add the stock, nutmeg, cinnamon, salt, and pepper. Bring to a boil, cover, and then lower the heat to a minimum and simmer for 12 minutes. Turn off the heat, remove the lid, cover the pan with a clean kitchen towel, and put the lid back on. Leave like that for about 5 minutes; this helps make the rice light and fluffy.

\Ing{4 tbsp olive oil}
\Ing{2 white onions, halved and thinly sliced}
Heat the olive oil in a large frying pan, add the onions, and sauté over medium heat for about 20 minutes, until dark brown. Transfer to paper towels to drain.

To serve, lightly break up the rice with a fork and then add the lentils and most of the onions, reserving a few for garnish. Taste for seasoning and adjust accordingly. Pile the rice high on a serving platter and top with the remaining onions. Serve hot, with the hot tomato sauce.
\end{recipe}
